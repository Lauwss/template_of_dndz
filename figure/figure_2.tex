\begin{tikzpicture}[node distance=2cm and 2cm, every node/.style={fill=white}, align=center]

  % 主流程节点
  \node (start) [startstop] {经典样本检测容量公式};
  \node (p2) [process, left=of start] {置信度临界值};
  \node (p1) [process, above  =of p2] {样本比例};
  \node (p3) [process, below  =of p2] {容忍区间};
  \node (p4) [process, above right=of start] {情形一样本容量};
  \node (p5) [process, below right=of start] {情形二样本容量};
  \begin{pgfonlayer}{background}
    \node[draw=blue,dashed, thick, inner sep=12pt, fit=(p1) (p3)] {};
  \end{pgfonlayer}
  
  \begin{pgfonlayer}{background}
    \node[draw=red, dashed, thick, inner sep=12pt, fit =(p4)(p5)] {};
  \end{pgfonlayer}
  % 分类文字单独放
  \node(text) [below=0.2cm of p3,text=blue] {\textbf{公式系数}};
  \node[below=0.2cm of p5,text=red] {\textbf{抽样次数}};
    \draw[<-,dashed] (start) |- node[midway]{标称值}(p1);
   \draw[<-,dashed] (start) -- (p2);
   \draw[<-,dashed] (start) |- node[midway]{经验值}(p3);
   \draw[->] (start) --++(2.25,0) --node[midway] {置信度为95\%}(p4);
   \draw[->] (start) --++(2.25,0) -- node[midway] {置信度为90\%}(p5);

  
\end{tikzpicture}