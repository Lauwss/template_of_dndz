\PassOptionsToPackage{quiet}{xeCJK}
\documentclass[withoutpreface,bwprint]{cumcmthesis}


\usepackage{longtable}  
\usepackage{geometry}
\geometry{left=2.5cm, right=2.5cm, top=2.5cm, bottom=2.5cm}

% 中文字号调整
\renewcommand{\rmdefault}{ptm}  % Times New Roman
\zihao{-4}  % 四号
\renewcommand{\arraystretch}{1.4}

% 表格与排版
\usepackage{tabularx}
\usepackage{booktabs}  % 三线表
\usepackage{makecell}  % \Xhline
\usepackage{diagbox}   % 斜线表头
\newcolumntype{C}{>{\centering\arraybackslash}X}
\newcolumntype{R}{>{\raggedleft\arraybackslash}X}
\newcolumntype{L}{>{\raggedright\arraybackslash}X}
\usepackage{etoolbox}
\BeforeBeginEnvironment{tabular}{\zihao{-5}}

% 数学公式
\usepackage{amsmath,amssymb}

% 文献管理
\usepackage[numbers,sort&compress]{natbib}  

% 链接、图形
\usepackage{url}
\usepackage{subcaption}  

% TikZ 绘图
\usepackage{tikz}
\usetikzlibrary{calc, arrows.meta, positioning, shapes.geometric, fit, backgrounds, decorations.text}

% 框架
\usepackage[framemethod=TikZ]{mdframed}

% 自定义颜色与命令
\definecolor{mygray}{RGB}{208,208,208}
\definecolor{mymagenta}{RGB}{226,150,116}
\newcommand*{\mytextstyle}{\sffamily\Large\bfseries\color{black!85}}

\newcommand{\arcarrow}[3]{%
   \pgfmathsetmacro{\rin}{1.7}
   \pgfmathsetmacro{\rmid}{2.2}
   \pgfmathsetmacro{\rout}{2.7}
   \pgfmathsetmacro{\astart}{#1}
   \pgfmathsetmacro{\aend}{#2}
   \pgfmathsetmacro{\atip}{5}
   \fill[mygray, very thick] (\astart+\atip:\rin)
                         arc (\astart+\atip:\aend:\rin)
      -- (\aend-\atip:\rmid)
      -- (\aend:\rout)   arc (\aend:\astart+\atip:\rout)
      -- (\astart:\rmid) -- cycle;
   \path[
      decoration = {
         text along path,
         text = {|\mytextstyle|#3},
         text align = {align = center},
         raise = -1.0ex
      },
      decorate
   ](\astart+\atip:\rmid) arc (\astart+\atip:\aend+\atip:\rmid);
}

\tikzset{
  >={Latex[width=2mm,length=2mm]},
  base/.style = {rectangle, rounded corners, draw=black,
                 minimum width=4cm, minimum height=1cm,
                 text centered, font=\sffamily},
  startstop/.style = {base, fill=red!30},
  decision/.style = {diamond, draw, text centered, inner sep=0pt,
                     minimum width=4cm, minimum height=1cm, aspect=2,
                     font=\sffamily},
  process/.style = {base, minimum width=3.5cm, fill=orange!15,
                    font=\ttfamily},
}

% 文档基本信息
\title{生产过程中的决策问题}
\tihao{}
\baominghao{}
\schoolname{}
\membera{}
\memberb{}
\memberc{}
\supervisor{}
\yearinput{}
\monthinput{}



%%%%%%%%%%%%%%%%%%%%%%%%%%%%%%%%%%%%%%%%%%%%%%%%%%%%%%%%%%%%%
%% 正文
\begin{document}

\maketitle
\thispagestyle{empty}
\begin{abstract}


本文聚焦企业电子产品生产流程中的核心决策问题,通过构建数学模型与量化分析,系统解决了抽样检测方案设计、单工序生产决策优化及多工序多零配件场景下的综合决策难题,为企业平衡成本与风险、提升运营效益提供了系统性解决方案。

\textbf{对于问题一},针对在 10\% 标称次品率下设计 95\% 与 90\% 信度下检测次数最少的抽样方案这一需求,我们基于统计假设检验理论构建模型。首先设定零假设(样本比例等于标称值)与备择假设(样本比例偏离标称值),利用中心极限定理将次品数量的分布近似为正态分布,推导样本容量计算方法。
\begin{table}[h!]
\centering
\begin{tabularx}{\textwidth}{XXXXX}
\Xhline{2pt}
\noalign{\vskip 1pt}
\toprule
置信水平 & 临界值 $Z_{\alpha/2}$ & 容忍区间 $d$ & 标称次品率 $p_0$ & 最小样本量 $n$ \\
\midrule
95\% & 1.960 & 0.02 & 0.10 & 865 \\
90\% & 1.645 & 0.02 & 0.10 & 609 \\
\bottomrule
\noalign{\vskip 1pt}
\Xhline{2pt}
\end{tabularx}
\end{table}

\textbf{对于问题二},基于六种零配件与成品参数组合,构建包含资源回收机制的多轮迭代利润模型,以最大化总利润为目标优化生产决策。模型采用\textbf{0-1 规划} 求解,考虑成品销售收益、检测成本、装配成本、调换损失及拆解后的零件循环利用价值。通过枚举法对 16 种可能策略组合进行计算。

\textbf{对于问题三},将模型推广至多道工序、多个零配件的通用场景,针对给定的 2 道工序、8 个零配件流程,构建包含半成品与零件双循环回收的利润模型。模型新增半成品检测与拆解决策,引入\textbf{遗传算法}解决多变量带来的计算复杂性。求解结果显示,最优策略为 “全零配件检测 + 全半成品检测 + 不检测成品 + 成品拆解 + 半成品 2 和 3 拆解”,此时单位利润最大(74.8344 元)。该策略通过早期检测降低次品流转风险,同时利用拆解回收高价值资源,实现了多环节成本与收益的动态平衡。

最后,本文分析了模型的优缺点:经验公式与0-1规划模型简洁高效,但存在对次品独立性假设的局限及变量增多时计算复杂度激增的问题,并提出结合成本整合优化与智能算法改进的方向,模型可推广至供应链质检、可靠性测试等多领域。

 

   \keywords{\textbf{抽样检测方案};\textbf{遗传算法};\textbf{0-1规划};\textbf{OC 曲线};\textbf{多工序生产}} 

\end{abstract}
%%%%%%%%%%%%%%%%%%%%%%%%%%%%%%%%%%%%%%%%%%%%%%%%%%%%%%%%%%%%% 

% \tableofcontents  % 目录
% \newpage

%%%%%%%%%%%%%%%%%%%%%%%%%%%%%%%%%%%%%%%%%%%%%%%%%%%%%%%%%%%%%  
\pagenumbering{arabic}
\section{问题重述}

\subsection{问题背景}
为高效利用有效资源、减少生产成本、提高运营收益,优化生产决策是每个企业经营管理中的核心环节。某企业生产电子产品需采购两种零配件,并将其装配后形成成品,且成品合格性受零配件质量影响,因此企业需通过抽样检测控制零配件次品率,并对不合格成品选择报废或拆解。生产过程中需决策是否检测零配件或成品、是否拆解不合格品,并考虑调换市场退回品的损失。对此我们需要设计抽样方案、优化生产阶段决策,并推广到多工序多零配件场景,以通过成本与风险平衡提升整体效益。


\subsection{题设数据}
\textbf{表一}给出了两种零配件和成品的次品率,以及购买单价、检测成本、市场售价等参数信息。

\textbf{表二}给出了八个零配件在两道工序中的次品率、购买单价、检测成本、拆解费用等参数信息。

\textbf{图一}给出了给出了2道工序、8个零配件的基本流程





\subsection{需要解决问题}

\textbf{问题一:}在$10\%$的标称度下,求出在$95\%$与$90\%$的信度下使检测次数尽可能少的抽样检测方案。

\textbf{问题二:}基于给定的六种情况下零配件与成品的参数,给出使利润最大化的最优生产决策方案。

\textbf{问题三:}在m道工序、n个零配件生产系统中,基于各环节的次品率、成本和售价等参数,制定最优的检测、拆解和调换决策方案,并对给定的2道工序、8个零配件给出具体决策依据和量化结果。


%%%%%%%%%%%%%%%%%%%%%%%%%%%%%%%%%%%%%%%%%%%%%%%%%%%%%%%%%%%%% 

\section{模型假设}
\textbf{假设一:}零配件的次品率相互独立,且与成品的次品率相互独立。

\textbf{假设二:}厂家每轮生产策略相同,即递归过程中零配件的次品率不变。

\textbf{假设三:}合格成品的市场售价与不合格成品的调换损失为定值,不随市场需求变化。

\textbf{假设四:}单位零配件所产生的期望利润相同。




%%%%%%%%%%%%%%%%%%%%%%%%%%%%%%%%%%%%%%%%%%%%%%%%%%%%%%%%%%%%% 

\newpage
\section{符号说明}
\begin{table}[H]
\begin{center}
 \begin{tabularx}{\textwidth}{XXX}
\Xhline{2pt}
\noalign{\vskip 1pt}
\toprule
符号    & 说明    & 单位 \\
\midrule
$n$ & 样本数 & - \\
$Z_{\alpha}$ & 临界值 & - \\
$\alpha$ & 置信度 & - \\
$p_0$ & 标称值 & - \\
$H_0$ & 零假设 & - \\
$H_1$ & 备择假设 & - \\
$X$ & 次品数量 & 个\\
$p$ & 次品率 & - \\
$d$ & 容忍区间 & - \\
$\Pi$ & 总利润期望 & 元 \\
$q$ & 合格率 & - \\
$c_{j}$ & 单位零件成本 & 元 \\
$s$ & 市场售价 & 元 \\
$x_{ij}$ & 第$i$道工序中第$j$个部件检测状态 & - \\
$d_{ij}$ & 第$i$道工序中第$j$个部件检测成本 & 个 \\
$a_{j}$ & 单位成品装配成本 & 元 \\
$t$ & 拆解成本 & 元 \\
$l$ & 调换损失 & 元 \\
$\theta $ & 回收比例 & - \\
\bottomrule
\noalign{\vskip 1pt}
\Xhline{2pt}
\end{tabularx}   
\end{center}
(其余符号详见正文)

\end{table}


%%%%%%%%%%%%%%%%%%%%%%%%%%%%%%%%%%%%%%%%%%%%%%%%%%%%%%%%%%%%% 
\newpage
\section{问题分析}
\subsection{对问题一的分析}

\begin{figure}[h]
   \centering
   \resizebox{0.5\linewidth}{!}{\begin{tikzpicture}[node distance=2cm and 2cm, every node/.style={fill=white}, align=center]
\def\layersep{4cm}

\tikzstyle{goal node}=[circle,fill=red!50,minimum size=18pt,inner sep=0pt]
\tikzstyle{criteria node}=[circle,fill=blue!50,minimum size=18pt,inner sep=0pt]
\tikzstyle{option node}=[circle,fill=green!50,minimum size=18pt,inner sep=0pt]
\tikzstyle{annot} = [text width=4em, text centered]

% 目标层
\node[goal node] (G) at (0, 0) {};
\node[below= 0.3cm of G] {期望利润};

% 准则层(成本构成3个方面)
\node[criteria node] (C1) at (\layersep, 2) {};
\node[below= 0.3cm of C1] {收入};

\node[criteria node] (C2) at (\layersep, -2) {};
\node[below= 0.3cm of C2] {成本};



% 目标->准则
\foreach \x in {1,2}
    \draw[->] (G) -- (C\x);

% 方案层(可以细化各成本细节)
\node[option node] (O0) at (2*\layersep, 3) {};
\node[below= 0.3cm of O0] {期望收入};

\node[option node] (O1) at (2*\layersep, 1) {};
\node[below= 0.5cm of O1] {零件检测成本};

\node[option node] (O2) at (2*\layersep, -1) {};
\node[below= 0.3cm of O2] {成品检测成本};

\node[option node] (O3) at (2*\layersep, -3) {};
\node[below= 0.3cm of O3] {次品拆解成本};   

\node[option node] (O4) at (2*\layersep, -5) {};
\node[below= 0.3cm of O4] {次品调换损失};


% 准则->方案(示意连接)
\draw[->] (C1) -- (O0);
\foreach \x in {1,2,3,4}
    \draw[->] (C2) -- (O\x);

% 层名注释
\node[annot] at (0, 4) {一级结构};
\node[annot] at (\layersep, 4) {二级结构};
\node[annot] at (2*\layersep, 4) {三级结构};


\end{tikzpicture}
}
   \caption{利润结构图}
   \label{fig:one}
\end{figure}


%%%%%%%%%%%%%%%%%%%%%%%%%%%%%%%%%%%%%%%%%%%%%%%%%%%%%%%%%%%%% 
\newpage
\newpage
\section{问题一的模型的建立和求解}
\subsection{模型建立}
\subsubsection{假设框架}
\subsection{模型求解}
将题设条件代入上式,同时,我们取容忍区间为0.02,最终求得结果如下表




%%%%%%%%%%%%%%%%%%%%%%%%%%%%%%%%%%%%%%%%%%%%%%%%%%%%%%%%%%%%% 
\section{问题二的模型的建立和求解}

%%%%%%%%%%%%%%%%%%%%%%%%%%%%%%%%%%%%%%%%%%%%%%%%%%%%%%%%%%%%% 
\section{问题三的模型的建立和求解}

\section{模型的评价}

\subsection{模型的优点}
\begin{itemize}[itemindent=2em]
\item 优点1:经验样本容量公式计算简便,无需复杂统计工具,并且很好的控制了两类错误,即生产方风险和使用方风险。在信度要求变化时,公式能灵活调整抽样方案,实现了动态调整。
、

\end{itemize}

\subsection{模型的缺点}
\begin{itemize}[itemindent=2em]
\item 缺点1:问题一假设样本中的次品出现是独立且概率恒定的,但实际生产中,次品可能集中在某些批次,此时抽样结果会低估真实风险。

\item 缺点2:0-1规划的计算复杂度高,变量增多时求解时间指数级增长,增大计算难度。


\end{itemize}
%%%%%%%%%%%%%%%%%%%%%%%%%%%%%%%%%%%%%%%%%%%%%%%%%%%%%%%%%%%%%

\section{模型的改进与推广}

\subsection{改进}
\begin{itemize}[itemindent=2em]
   \item 改进1:问题一可进行成本整合优化:将公式嵌入决策树或线性规划,同时优化样本量、检测成本、拆解费用等。
   \item 改进2:
\end{itemize}


\subsection{推广}
\begin{itemize}[itemindent=2em]
    \item 推广1:经验样本容量公式可推广到供应商来料检验、电子产品可靠性测试与寿命评估、市场质量反馈与售后分析等实际问题。
   \item 推广2:0-1规划可推广到投资组合优化、电路设计、广告投放、网络与路径优化等所有二元决策类问题。
\end{itemize}
\newpage
%%%%%%%%%%%%%%%%%%%%%%%%%%%%%%%%%%%%%%%%%%%%%%%%%%%%%%%%%%%%%
%% 参考文献
\nocite{*}
\bibliographystyle{gbt7714-numerical}  % 引用格式
\bibliography{ref.bib}  % bib源

\newpage
%%%%%%%%%%%%%%%%%%%%%%%%%%%%%%%%%%%%%%%%%%%%%%%%%%%%%%%%%%%%%
%% 附录
\begin{appendices}
\section{文件列表}
\begin{table}[H]
\centering
\begin{tabularx}{\textwidth}{LL}
\toprule
文件名   & 功能描述 \\
\midrule
 exercise\_1.py & 问题一程序代码 \\
 exercise\_2.py & 问题二程序代码 \\
 exercise\_3.py & 问题三程序代码 \\

\bottomrule
\end{tabularx}
\label{tab:文件列表}
\end{table}

\section{代码}
\noindent 
 exercise\_1.py
\lstinputlisting[language=python]{code/exercise_1.py}
exercise\_2.py
\lstinputlisting[language=python]{code/exercise_2.py}
exercise\_3.py
 \lstinputlisting[language=python]{code/exercise_3.py}


\end{appendices}
\end{document}


%%%%%双图模板%%%%%%
% \begin{figure}
% \centering
% \subcaptionbox{炉温曲线示意图\label{fig:双图a}}
% {\includegraphics[width=.4\textwidth]{炉温曲线示意图.png}}
% \subcaptionbox{问题1炉温曲线\label{fig:双图b}}
% {\includegraphics[width=.4\textwidth]{问题1炉温曲线.png}}
% \caption{双图}\label{fig:双图}
% \end{figure} 
%%%%%双图模板%%%%%%