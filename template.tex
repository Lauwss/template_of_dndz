\PassOptionsToPackage{quiet}{xeCJK}
\documentclass[withoutpreface,bwprint]{cumcmthesis}


\usepackage{geometry}
\geometry{left=2.5cm, right=2.5cm, top=2.5cm, bottom=2.5cm}

% 中文字号调整
\renewcommand{\rmdefault}{ptm}  % Times New Roman
\zihao{-4}  % 四号
\renewcommand{\arraystretch}{1.4}

% 表格与排版
\usepackage{tabularx}
\usepackage{booktabs}  % 三线表
\usepackage{makecell}  % \Xhline
\usepackage{diagbox}   % 斜线表头
\newcolumntype{C}{>{\centering\arraybackslash}X}
\newcolumntype{R}{>{\raggedleft\arraybackslash}X}
\newcolumntype{L}{>{\raggedright\arraybackslash}X}
\usepackage{etoolbox}
\BeforeBeginEnvironment{tabular}{\zihao{-5}}

% 数学公式
\usepackage{amsmath,amssymb}

% 文献管理
\usepackage[numbers,sort&compress]{natbib}  

% 链接、图形
\usepackage{url}
\usepackage{subcaption}  

% TikZ 绘图
\usepackage{tikz}
\usetikzlibrary{calc, arrows.meta, positioning, shapes.geometric, fit, backgrounds, decorations.text}

% 框架
\usepackage[framemethod=TikZ]{mdframed}

% 自定义颜色与命令
\definecolor{mygray}{RGB}{208,208,208}
\definecolor{mymagenta}{RGB}{226,150,116}
\newcommand*{\mytextstyle}{\sffamily\Large\bfseries\color{black!85}}

\newcommand{\arcarrow}[3]{%
   \pgfmathsetmacro{\rin}{1.7}
   \pgfmathsetmacro{\rmid}{2.2}
   \pgfmathsetmacro{\rout}{2.7}
   \pgfmathsetmacro{\astart}{#1}
   \pgfmathsetmacro{\aend}{#2}
   \pgfmathsetmacro{\atip}{5}
   \fill[mygray, very thick] (\astart+\atip:\rin)
                         arc (\astart+\atip:\aend:\rin)
      -- (\aend-\atip:\rmid)
      -- (\aend:\rout)   arc (\aend:\astart+\atip:\rout)
      -- (\astart:\rmid) -- cycle;
   \path[
      decoration = {
         text along path,
         text = {|\mytextstyle|#3},
         text align = {align = center},
         raise = -1.0ex
      },
      decorate
   ](\astart+\atip:\rmid) arc (\astart+\atip:\aend+\atip:\rmid);
}

\tikzset{
  >={Latex[width=2mm,length=2mm]},
  base/.style = {rectangle, rounded corners, draw=black,
                 minimum width=4cm, minimum height=1cm,
                 text centered, font=\sffamily},
  startstop/.style = {base, fill=red!30},
  decision/.style = {diamond, draw, text centered, inner sep=0pt,
                     minimum width=4cm, minimum height=1cm, aspect=2,
                     font=\sffamily},
  process/.style = {base, minimum width=3.5cm, fill=orange!15,
                    font=\ttfamily},
}

% 文档基本信息
\title{生产过程中的决策问题}
\tihao{}
\baominghao{}
\schoolname{}
\membera{}
\memberb{}
\memberc{}
\supervisor{}
\yearinput{}
\monthinput{}



%%%%%%%%%%%%%%%%%%%%%%%%%%%%%%%%%%%%%%%%%%%%%%%%%%%%%%%%%%%%%
%% 正文
\begin{document}

\maketitle
\thispagestyle{empty}
\begin{abstract}



\textbf{对于问题一,}


   \keywords{} 

\end{abstract}
%%%%%%%%%%%%%%%%%%%%%%%%%%%%%%%%%%%%%%%%%%%%%%%%%%%%%%%%%%%%% 

% \tableofcontents  % 目录
% \newpage

%%%%%%%%%%%%%%%%%%%%%%%%%%%%%%%%%%%%%%%%%%%%%%%%%%%%%%%%%%%%%  
\pagenumbering{arabic}
\section{问题重述}

\subsection{问题背景}



\subsection{题设数据}






\subsection{需要解决问题}

\textbf{问题一:}


%%%%%%%%%%%%%%%%%%%%%%%%%%%%%%%%%%%%%%%%%%%%%%%%%%%%%%%%%%%%% 

\section{模型假设}
\textbf{假设一:}




%%%%%%%%%%%%%%%%%%%%%%%%%%%%%%%%%%%%%%%%%%%%%%%%%%%%%%%%%%%%% 

\newpage
\section{符号说明}
\begin{table}[H]
\begin{center}
 \begin{tabularx}{\textwidth}{XXX}
\Xhline{2pt}
\noalign{\vskip 1pt}
\toprule
符号    & 说明    & 单位 \\
\midrule
$n$ & 样本数 & - \\
$Z_{\alpha}$ & 临界值 & - \\
$\alpha$ & 置信度 & - \\
$p_0$ & 标称值 & - \\
$H_0$ & 零假设 & - \\
$H_1$ & 备择假设 & - \\
$X$ & 次品数量 & - \\
$p$ & 次品率 & - \\
\bottomrule
\noalign{\vskip 1pt}
\Xhline{2pt}
\end{tabularx}   
\end{center}
(其余符号详见正文)

\end{table}


%%%%%%%%%%%%%%%%%%%%%%%%%%%%%%%%%%%%%%%%%%%%%%%%%%%%%%%%%%%%% 
\newpage
\section{问题分析}
\subsection{对问题一的分析}
题目要求我们在标称值确定的情况下,设计出一种合理的抽样检测方案,使得在给定的两种不同情形下,抽样的次数最小。要使抽样的次数最小,即使得抽样的样本容量最小即可。基于此,我们根据经典检验样本容量公式,结合题目的标称值得到理想样本比例,同时根据不同情形下的置信度确定临界值,最终计算出抽样次数。
\begin{figure}[h]
   \centering
   \resizebox{0.8\linewidth}{!}{\begin{tikzpicture}[node distance=2cm and 2cm, every node/.style={fill=white}, align=center]

  % 主流程节点
  \node (start) [startstop] {经典样本检测容量公式};
  \node (p2) [process, left=of start] {置信度临界值};
  \node (p1) [process, above  =of p2] {样本比例};
  \node (p3) [process, below  =of p2] {容忍区间};
  \node (p4) [process, above right=of start] {情形一样本容量};
  \node (p5) [process, below right=of start] {情形二样本容量};
  \begin{pgfonlayer}{background}
    \node[draw=blue,dashed, thick, inner sep=12pt, fit=(p1) (p3)] {};
  \end{pgfonlayer}
  
  \begin{pgfonlayer}{background}
    \node[draw=red, dashed, thick, inner sep=12pt, fit =(p4)(p5)] {};
  \end{pgfonlayer}
  % 分类文字单独放
  \node(text) [below=0.2cm of p3,text=blue] {\textbf{公式系数}};
  \node[below=0.2cm of p5,text=red] {\textbf{抽样次数}};
    \draw[<-,dashed] (start) |- node[midway]{标称值}(p1);
   \draw[<-,dashed] (start) -- (p2);
   \draw[<-,dashed] (start) |- node[midway]{经验值}(p3);
   \draw[->] (start) --++(2.25,0) --node[midway] {置信度为95\%}(p4);
   \draw[->] (start) --++(2.25,0) -- node[midway] {置信度为90\%}(p5);

  
\end{tikzpicture}}
   \caption{第一问流程图}
   \label{fig:one}
\end{figure}


%%%%%%%%%%%%%%%%%%%%%%%%%%%%%%%%%%%%%%%%%%%%%%%%%%%%%%%%%%%%% 

\section{问题一的模型的建立和求解}
\subsection{模型建立}
\subsubsection{假设框架}
根据题设条件,我们构建了如下假设框架:
\begin{itemize}
   \item 零假设$H_0$:样本比例$p$等于标称值$p_0$
   \item 备择假设$H_1$:对于拒收情形,样本比例$p$大于标称值$p_0$;对于接受情形,样本比例$p$小于标称值$p_0$。
\end{itemize}
\subsubsection{样本容量公式}
假设从总体中抽取容量为$n$的样本,记其中的次品个数为随机变量$X$,显然该变量服从二项分布
\begin{equation}
X\sim B(n,p)
\end{equation}
其中$p$为样本比例。根据中心极限定理,当$n$足够大时,$X$可以近似服从正态分布,即\begin{equation}
X\sim N(np,np(1-p))
\end{equation}
进一步的,我们可确定样本中的次品率的统计量$\hat{p}$近似服从正态分布,即
\begin{equation}
\hat{p}\sim N(p,\frac{p(1-p)}{n})
\end{equation}
由上式我们可推导出经典样本检测容量公式
\begin{equation}
n=\frac{Z^2_{\alpha}p(1-p)}{d^2}
\end{equation}
其中$Z_{\alpha}$为置信区间在标准正态分布中的临界值,$\alpha$为置信度,$d$为容忍区间,一般取经验值。

\subsection{模型求解}
将题设条件代入上式,同时,我们取容忍区间为0.02,最终求得结果如下表
\subsection{结果分析}





%%%%%%%%%%%%%%%%%%%%%%%%%%%%%%%%%%%%%%%%%%%%%%%%%%%%%%%%%%%%% 




%%%%%%%%%%%%%%%%%%%%%%%%%%%%%%%%%%%%%%%%%%%%%%%%%%%%%%%%%%%%% 

\section{模型的评价}

\subsection{模型的优点}
\begin{itemize}[itemindent=2em]
\item 优点1:
\end{itemize}

\subsection{模型的缺点}
\begin{itemize}[itemindent=2em]
\item 缺点1:
\end{itemize}
%%%%%%%%%%%%%%%%%%%%%%%%%%%%%%%%%%%%%%%%%%%%%%%%%%%%%%%%%%%%%

\section{模型的改进与推广}

\subsection{改进}
\begin{itemize}[itemindent=2em]
   \item 改进1:
   \item 改进2:
\end{itemize}


\subsection{推广}
\begin{itemize}[itemindent=2em]
    \item 推广1:
   \item 推广2:
\end{itemize}
\newpage
%%%%%%%%%%%%%%%%%%%%%%%%%%%%%%%%%%%%%%%%%%%%%%%%%%%%%%%%%%%%%
%% 参考文献
\nocite{*}
\bibliographystyle{gbt7714-numerical}  % 引用格式
\bibliography{ref.bib}  % bib源

\newpage
%%%%%%%%%%%%%%%%%%%%%%%%%%%%%%%%%%%%%%%%%%%%%%%%%%%%%%%%%%%%%
%% 附录
\begin{appendices}
\section{文件列表}
\begin{table}[H]
\centering
\begin{tabularx}{\textwidth}{LL}
\toprule
文件名   & 功能描述 \\
\midrule
 第一问代码.py&问题一程序代码\\
第二问.py & 问题二程序代码 \\
第三及第四问代码.py&问题三与问题四程序代码\\

\bottomrule
\end{tabularx}
\label{tab:文件列表}
\end{table}

\section{代码}
\noindent 
%  第一问代码.py
% \lstinputlisting[language=python]{code/第一问代码.py}
% 第二问.py
% \lstinputlisting[language=python]{code/第二问.py}
% 第三及第四问代码.py
%  \lstinputlisting[language=lingo]{code/第三及第四问代码.py}


\end{appendices}
\end{document}


%%%%%双图模板%%%%%%
% \begin{figure}
% \centering
% \subcaptionbox{炉温曲线示意图\label{fig:双图a}}
% {\includegraphics[width=.4\textwidth]{炉温曲线示意图.png}}
% \subcaptionbox{问题1炉温曲线\label{fig:双图b}}
% {\includegraphics[width=.4\textwidth]{问题1炉温曲线.png}}
% \caption{双图}\label{fig:双图}
% \end{figure} 
%%%%%双图模板%%%%%%