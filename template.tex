\PassOptionsToPackage{quiet}{xeCJK}
\documentclass[withoutpreface,bwprint]{cumcmthesis}

\usepackage{geometry}
\geometry{left=2.5cm, right=2.5cm, top=2.5cm, bottom=2.5cm}

% 中文字号调整
\renewcommand{\rmdefault}{ptm}  % Times New Roman
\zihao{-4}  % 四号
\renewcommand{\arraystretch}{1.4}

% 表格与排版
\usepackage{tabularx}
\usepackage{booktabs}  % 三线表
\usepackage{makecell}  % \Xhline
\usepackage{diagbox}   % 斜线表头
\newcolumntype{C}{>{\centering\arraybackslash}X}
\newcolumntype{R}{>{\raggedleft\arraybackslash}X}
\newcolumntype{L}{>{\raggedright\arraybackslash}X}
\usepackage{etoolbox}
\BeforeBeginEnvironment{tabular}{\zihao{-5}}

% 数学公式
\usepackage{amsmath,amssymb}

% 文献管理
\usepackage[numbers,sort&compress]{natbib}  

% 链接、图形
\usepackage{url}
\usepackage{subcaption}  

% TikZ 绘图
\usepackage{tikz}
\usetikzlibrary{calc, arrows.meta, positioning, shapes.geometric, fit, backgrounds, decorations.text}

% 框架
\usepackage[framemethod=TikZ]{mdframed}

% 自定义颜色与命令
\definecolor{mygray}{RGB}{208,208,208}
\definecolor{mymagenta}{RGB}{226,150,116}
\newcommand*{\mytextstyle}{\sffamily\Large\bfseries\color{black!85}}

\newcommand{\arcarrow}[3]{%
   \pgfmathsetmacro{\rin}{1.7}
   \pgfmathsetmacro{\rmid}{2.2}
   \pgfmathsetmacro{\rout}{2.7}
   \pgfmathsetmacro{\astart}{#1}
   \pgfmathsetmacro{\aend}{#2}
   \pgfmathsetmacro{\atip}{5}
   \fill[mygray, very thick] (\astart+\atip:\rin)
                         arc (\astart+\atip:\aend:\rin)
      -- (\aend-\atip:\rmid)
      -- (\aend:\rout)   arc (\aend:\astart+\atip:\rout)
      -- (\astart:\rmid) -- cycle;
   \path[
      decoration = {
         text along path,
         text = {|\mytextstyle|#3},
         text align = {align = center},
         raise = -1.0ex
      },
      decorate
   ](\astart+\atip:\rmid) arc (\astart+\atip:\aend+\atip:\rmid);
}

\tikzset{
  >={Latex[width=2mm,length=2mm]},
  base/.style = {rectangle, rounded corners, draw=black,
                 minimum width=4cm, minimum height=1cm,
                 text centered, font=\sffamily},
  startstop/.style = {base, fill=red!30},
  decision/.style = {diamond, draw, text centered, inner sep=0pt,
                     minimum width=4cm, minimum height=1cm, aspect=2,
                     font=\sffamily},
  process/.style = {base, minimum width=3.5cm, fill=orange!15,
                    font=\ttfamily},
}

% 文档基本信息
\title{长江水质的综合评价与预测}
\tihao{}
\baominghao{}
\schoolname{}
\membera{}
\memberb{}
\memberc{}
\supervisor{}
\yearinput{}
\monthinput{}



%%%%%%%%%%%%%%%%%%%%%%%%%%%%%%%%%%%%%%%%%%%%%%%%%%%%%%%%%%%%%
%% 正文
\begin{document}

\maketitle
\thispagestyle{empty}
\begin{abstract}



\textbf{对于问题一,}


   \keywords{} 

\end{abstract}
%%%%%%%%%%%%%%%%%%%%%%%%%%%%%%%%%%%%%%%%%%%%%%%%%%%%%%%%%%%%% 

% \tableofcontents  % 目录
% \newpage

%%%%%%%%%%%%%%%%%%%%%%%%%%%%%%%%%%%%%%%%%%%%%%%%%%%%%%%%%%%%%  
\pagenumbering{arabic}
\section{问题重述}

\subsection{问题背景}



\subsection{题设数据}






\subsection{需要解决问题}

\textbf{问题一:}


%%%%%%%%%%%%%%%%%%%%%%%%%%%%%%%%%%%%%%%%%%%%%%%%%%%%%%%%%%%%% 

\section{模型假设}
\textbf{假设一:}




%%%%%%%%%%%%%%%%%%%%%%%%%%%%%%%%%%%%%%%%%%%%%%%%%%%%%%%%%%%%% 

\newpage
\section{符号说明}
\begin{table}[H]
\begin{center}
 \begin{tabularx}{\textwidth}{XXX}
\Xhline{2pt}
\noalign{\vskip 1pt}
\toprule
符号    & 说明    & 单位 \\
\midrule
$x_{ij}$   & 第$i$个月的第$j$项指标值 &$mg/L$ \\
$\tilde{x}_{ij}$     & 第$i$个月第$j$项指标的标准化值 & - \\
$r_{ij}$     & 第$i$个月第$j$项指标的倒数 & $L/mg$\\
$\boldsymbol{A}_{4\times4}$     &准则—目标判断矩阵& -\\
$\boldsymbol{B}_{k}$     &方案—准则判断矩阵& -\\
$s_{ij}^k$     &第$i$个观测点的$k$项指标与第$j$个观测点的$k$项指标比值& -\\
$\mu_k$   & 第$k$项指标I类水质的标准值 &$mg/L$ \\
$\sigma_k$   & 第$k$项指标的变权参数 & \\
$w_k$&第$k$项指标的变权函数 &\\
$L_{IV}^k$&第$k$项指标的IV类水质标准值&$mg/L$\\
$W$   &水质综合评价函数 & \\
$E$   &扩散系数 & \\
$S$   &外部污染源输入项 & \\

\bottomrule
\noalign{\vskip 1pt}
\Xhline{2pt}
\end{tabularx}   
\end{center}
(其余符号详见正文)

\end{table}


%%%%%%%%%%%%%%%%%%%%%%%%%%%%%%%%%%%%%%%%%%%%%%%%%%%%%%%%%%%%% 

\section{问题分析}
\subsection{对问题一的分析}

%%%%%%%%%%%%%%%%%%%%%%%%%%%%%%%%%%%%%%%%%%%%%%%%%%%%%%%%%%%%% 

\section{问题一的模型的建立和求解}
\subsection{模型建立}


\subsection{模型求解}

\subsection{结果分析}





%%%%%%%%%%%%%%%%%%%%%%%%%%%%%%%%%%%%%%%%%%%%%%%%%%%%%%%%%%%%% 




%%%%%%%%%%%%%%%%%%%%%%%%%%%%%%%%%%%%%%%%%%%%%%%%%%%%%%%%%%%%% 

\section{模型的评价}

\subsection{模型的优点}
\begin{itemize}[itemindent=2em]
\item 优点1:
\end{itemize}

\subsection{模型的缺点}
\begin{itemize}[itemindent=2em]
\item 缺点1:
\end{itemize}
%%%%%%%%%%%%%%%%%%%%%%%%%%%%%%%%%%%%%%%%%%%%%%%%%%%%%%%%%%%%%

\section{模型的改进与推广}

\subsection{改进}
\begin{itemize}[itemindent=2em]
   \item 改进1:
   \item 改进2:
\end{itemize}


\subsection{推广}
\begin{itemize}[itemindent=2em]
    \item 推广1:
   \item 推广2:
\end{itemize}
\newpage
%%%%%%%%%%%%%%%%%%%%%%%%%%%%%%%%%%%%%%%%%%%%%%%%%%%%%%%%%%%%%
%% 参考文献
\nocite{*}
\bibliographystyle{gbt7714-numerical}  % 引用格式
\bibliography{ref.bib}  % bib源

\newpage
%%%%%%%%%%%%%%%%%%%%%%%%%%%%%%%%%%%%%%%%%%%%%%%%%%%%%%%%%%%%%
%% 附录
\begin{appendices}
\section{文件列表}
\begin{table}[H]
\centering
\begin{tabularx}{\textwidth}{LL}
\toprule
文件名   & 功能描述 \\
\midrule
 第一问代码.py&问题一程序代码\\
第二问.py & 问题二程序代码 \\
第三及第四问代码.py&问题三与问题四程序代码\\

\bottomrule
\end{tabularx}
\label{tab:文件列表}
\end{table}

\section{代码}
\noindent 
%  第一问代码.py
% \lstinputlisting[language=python]{code/第一问代码.py}
% 第二问.py
% \lstinputlisting[language=python]{code/第二问.py}
% 第三及第四问代码.py
%  \lstinputlisting[language=lingo]{code/第三及第四问代码.py}


\end{appendices}
\end{document}


%%%%%双图模板%%%%%%
\begin{figure}
\centering
\subcaptionbox{炉温曲线示意图\label{fig:双图a}}
{\includegraphics[width=.4\textwidth]{炉温曲线示意图.png}}
\subcaptionbox{问题1炉温曲线\label{fig:双图b}}
{\includegraphics[width=.4\textwidth]{问题1炉温曲线.png}}
\caption{双图}\label{fig:双图}
\end{figure} 
%%%%%双图模板%%%%%%